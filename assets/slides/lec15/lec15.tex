\PassOptionsToPackage{unicode=true}{hyperref} % options for packages loaded elsewhere
\PassOptionsToPackage{hyphens}{url}
%
\documentclass[ignorenonframetext,]{beamer}
\usepackage{pgfpages}
\setbeamertemplate{caption}[numbered]
\setbeamertemplate{caption label separator}{: }
\setbeamercolor{caption name}{fg=normal text.fg}
\beamertemplatenavigationsymbolsempty
% Prevent slide breaks in the middle of a paragraph:
\widowpenalties 1 10000
\raggedbottom
\setbeamertemplate{part page}{
\centering
\begin{beamercolorbox}[sep=16pt,center]{part title}
  \usebeamerfont{part title}\insertpart\par
\end{beamercolorbox}
}
\setbeamertemplate{section page}{
\centering
\begin{beamercolorbox}[sep=12pt,center]{part title}
  \usebeamerfont{section title}\insertsection\par
\end{beamercolorbox}
}
\setbeamertemplate{subsection page}{
\centering
\begin{beamercolorbox}[sep=8pt,center]{part title}
  \usebeamerfont{subsection title}\insertsubsection\par
\end{beamercolorbox}
}
\AtBeginPart{
  \frame{\partpage}
}
\AtBeginSection{
  \ifbibliography
  \else
    \frame{\sectionpage}
  \fi
}
\AtBeginSubsection{
  \frame{\subsectionpage}
}
\usepackage{lmodern}
\usepackage{amssymb,amsmath}
\usepackage{ifxetex,ifluatex}
\usepackage{fixltx2e} % provides \textsubscript
\ifnum 0\ifxetex 1\fi\ifluatex 1\fi=0 % if pdftex
  \usepackage[T1]{fontenc}
  \usepackage[utf8]{inputenc}
  \usepackage{textcomp} % provides euro and other symbols
\else % if luatex or xelatex
  \usepackage{unicode-math}
  \defaultfontfeatures{Ligatures=TeX,Scale=MatchLowercase}
\fi
% use upquote if available, for straight quotes in verbatim environments
\IfFileExists{upquote.sty}{\usepackage{upquote}}{}
% use microtype if available
\IfFileExists{microtype.sty}{%
\usepackage[]{microtype}
\UseMicrotypeSet[protrusion]{basicmath} % disable protrusion for tt fonts
}{}
\IfFileExists{parskip.sty}{%
\usepackage{parskip}
}{% else
\setlength{\parindent}{0pt}
\setlength{\parskip}{6pt plus 2pt minus 1pt}
}
\usepackage{hyperref}
\hypersetup{
            pdftitle={Extentions to Models for Count Data},
            pdfauthor={Nicholas Reich, transcribed by Herb Susmann edited by Hachem Saddiki},
            pdfborder={0 0 0},
            breaklinks=true}
\urlstyle{same}  % don't use monospace font for urls
\newif\ifbibliography
\setlength{\emergencystretch}{3em}  % prevent overfull lines
\providecommand{\tightlist}{%
  \setlength{\itemsep}{0pt}\setlength{\parskip}{0pt}}
\setcounter{secnumdepth}{0}

% set default figure placement to htbp
\makeatletter
\def\fps@figure{htbp}
\makeatother

%       ************************************************
%       **        LaTeX preamble to be used with all 
%	**        statsTeachR labs/handouts.
%
%	Author: Nicholas G Reich
%	Last modified: July 2017
%	************************************************

%\documentclass[table]{beamer}

%	Set theme (a nice plain one)
\usetheme{Malmoe}

%	Use named colors, set main color of theme
%		to match Web site color:
\definecolor{MainColor}{RGB}{10, 74, 109}
\colorlet{MainColorMedium}{MainColor!50}
\colorlet{MainColorLight}{MainColor!20}
\usecolortheme[named=MainColor]{structure} 

%	For tables
%[dvipsnames] [table]
\usepackage{xcolor}

%% calling tabu.sty, assuming a particular directory structure
\usepackage{../../slide-includes/tabu}	% Even fancier than tabulary
\usepackage{multirow}

%	Just for the degree symbol
\usepackage{textcomp}

%	Get rid of footline (page, author, etc. on each slide)
\setbeamertemplate{footline}{}
%	Get rid of navigation buttons
\setbeamertemplate{navigation symbols}{}

%	Make footnotes not ugly
\usepackage{hanging}
\setbeamertemplate{footnote}{\raggedright\hangpara{1em}{1}\makebox[1em][l]{\insertfootnotemark}\footnotesize\insertfootnotetext\par}

%	Text style for code snippets inline in text:
\newcommand{\codeInline}[1]{\texttt{#1}}

%	Text style for emphasis stronger than \emph:
%		(Note, this doesn't toggle the way \emph does.
%			(Note, this can be done, didn't seem worth the trouble.))
\newcommand{\strong}[1]{{\bfseries{#1}}}


%        ******	Define title page	**********************
\setbeamertemplate{title page}{
	{\color{MainColor}
	% There must be a better way than this -vspace at
	%	 the top and bottom of the page to reduce the 
	%	 bottom margin, but I can't find one that works.
	%\vspace{-6em}

% 	% Go to a lot of trouble to get the title in a
% 	%	nice box, since customizing a beamer block
% 	%	does not entirely work here (I don't know why)
	\newlength{\titleBoxWidth}
	\setlength{\titleBoxWidth}{\textwidth}
	\addtolength{\titleBoxWidth}{-2.0em}
	\setlength{\fboxsep}{1.0em}
	\setlength{\fboxrule}{0pt}
	\fcolorbox{MainColor!25}{MainColor!25}{
		\parbox{\titleBoxWidth}{
			\raggedright
			\LARGE\textbf{\inserttitle}
		}	% end parbox
	}	% end fcolorbox

	\vfill
	\small{Author: \insertauthor}
	\vspace{\baselineskip}

	\small{Course: \underline{\href{http://nickreich.github.io/cda}{Categorical Data Analysis}} (BIOSTATS 743)}

%	\small{\Instructor}
%	\vspace{\baselineskip}

%	\small{\emph{This material is part of the \strong{statsTeachR} project}}

	\vfill
	
	\tiny{\emph{Made available under the \underline{\href{http://creativecommons.org/licenses/by-sa/4.0/}{Creative Commons Attribution-ShareAlike 4.0 International License}.}} \hfill \includegraphics[height=1em]{../../slide-includes/by-sa-compact.png}
 }


		\vspace{-15em}

	}	% end color
	\clearpage
}	% end define title page

\input{../../slide-includes/shortcuts}

\hypersetup{colorlinks,linkcolor=,urlcolor=MainColor}
\def\begincols{\begin{columns}}
\def\begincol{\begin{column}}
\def\endcol{\end{column}}
\def\endcols{\end{columns}}

\title{Extentions to Models for Count Data}
\author{Nicholas Reich, transcribed by Herb Susmann edited by Hachem Saddiki}
\date{}

\begin{document}
\frame{\titlepage}

\begin{frame}{Extensions to Models for Count Data}
\protect\hypertarget{extensions-to-models-for-count-data}{}

There are several ways to extend models for count data in order to
capture properties like overdispersion.

\begin{itemize}
\tightlist
\item
  Poisson model with adjustment for overdispersion (see previous notes)
\item
  Poisson-Gamma Model
\item
  Generalized Linear Mixed Models (GLMMs)
\end{itemize}

\end{frame}

\begin{frame}{Poisson-Gamma Model}
\protect\hypertarget{poisson-gamma-model}{}

A Poisson-Gamma model is one way to account for overdispersion in models
of count data. The model has two parts:

\begin{itemize}
\tightlist
\item
  First, we assume that the outcome variable follows a Poisson
  distribution: \(Y | \lambda \sim \mathrm{Poisson}(\lambda)\).
\item
  Second, we assume that the rate parameter for that Poisson
  distribution itself follows a Gamma distribution:
  \(\lambda \sim \mathrm{Gamma}(k, \mu)\).

  \begin{itemize}
  \tightlist
  \item
    Under this parameterization, \(\mathrm{E}[\lambda] = \mu\),
    \(\mathrm{Var}[\lambda] = \mu^2/k\).
  \item
    We can alternatively parameterize in terms of a dispersion parameter
    \(\gamma = 1/k\).
  \end{itemize}
\end{itemize}

\end{frame}

\begin{frame}{Poisson-Gamma Model}
\protect\hypertarget{poisson-gamma-model-1}{}

\begin{itemize}
\item
  Under these two assumptions, the marginal distribution of \(Y\)
  follows a negative binomial distribution:
  \(Y \sim \mathrm{NegativeBinomial(k,\mu)}\)
\item
  See
  \href{https://probabilityandstats.wordpress.com/tag/poisson-gamma-mixture/}{this
  blog post} for a proof that the Poisson-Gamma model is a negative
  binomial distribution.
\item
  The mean and variance of the Poisson-Gamma model is not equal (as
  opposed to a Poisson model), which allows it to account for
  overdispersion.
\end{itemize}

\end{frame}

\begin{frame}{Poisson-Gamma Model}
\protect\hypertarget{poisson-gamma-model-2}{}

\begin{itemize}
\tightlist
\item
  The expected value of \(Y\) is given by: \[
  \begin{aligned}
  \mathrm{E}[Y] &= \mathrm{E}[\mathrm{E}[Y|\lambda]] \\
              &= \mathrm{E}[\lambda] \\
              &= \mu \\
  \end{aligned}
  \]
\item
  And the variance: \[
  \begin{aligned}
  \mathrm{Var}[Y] &= \mathrm{E}[\mathrm{Var}[Y|\lambda]] + \mathrm{Var}[\mathrm{E}(Y|\lambda)] \\
                &= \mathrm{E}[\lambda] + \mathrm{Var}[\lambda] \\
                &= \mu + \mu^2/k\text{, or equivalently} \\
                &= \mu + \gamma \mu^2
  \end{aligned}
  \]
\item
  Note that as \(\gamma \rightarrow 0\) (\(k \rightarrow \infty\)), the
  distribution of \(Y\) approaches a Poisson distribution.
\end{itemize}

\end{frame}

\begin{frame}{Generalized Linear Mixed Models}
\protect\hypertarget{generalized-linear-mixed-models}{}

Another approach is to use a Generalized Linear Mixed Model (GLMM).

\begin{itemize}
\item
  First, assume that the outcome \(Y_i\) follows a Poisson distribution.
\item
  Assume the link-transformed expected value of the outcome is a linear
  function of the covariates and random effects: \[
  \begin{aligned}
  \log(\mathrm{E}[Y_i|\mu_i]) &= X_{ij}^T\beta + \mu_i \\
  \end{aligned}
  \]
\item
  Finally, assume that the random effects \(u_i\) follow a distribution:
  \[
  u_i \sim N(0,\sigma^2)
  \]
\item
  This example uses a log link and assumes the \(u_i\) are normally
  distributed.
\end{itemize}

\end{frame}

\begin{frame}{Generalized Linear Mixed Models}
\protect\hypertarget{generalized-linear-mixed-models-1}{}

\begin{itemize}
\tightlist
\item
  Note that you need to use a link that transforms the linear predictor
  to a non-negative value. For example, the identity link leads to
  structural problems because a negative linear predictor implies a
  negative expected count, which is impossible.
\end{itemize}

Other choices are possible for the distribution of \(u_i\):

\begin{itemize}
\tightlist
\item
  Assuming \(u_i \sim \mathrm{Gamma}(1,\gamma)\) implies a negative
  binomially distributed outcome \(Y\).
\item
  Another possible choice is assume \(u_i\) follow a log-normal
  distribution.
\item
  Each choice implies a different structure for the random intercepts.
\end{itemize}

\end{frame}

\end{document}
